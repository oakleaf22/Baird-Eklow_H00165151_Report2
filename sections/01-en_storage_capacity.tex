\section{Question 1}



Before we can calculate the energy storage capacity that Scotland requires to satisfy its electricity demand at all times with renewable generation, we need to:
\begin{itemize}
	\item Determine Scotland's current typical electricity demand
	\item Upscale the current generation mix for renewable technologies
	\item Calculate the necessary energy storage power capacity
\end{itemize}



%%% SUBSECTION %%%

\subsection{Typical electricity demand}

The Department for Business, Energy {\&} Industrial Strategy (BEIS) of the UK Government publishes electricity generation and supply statistics on a quarterly basis \citep{BEIS2018ElecUK}.
Table~\ref{tbl:elec_demand} presents the most up-to-date figures for Scotland.
England is a regular importer of electricity from Scotland \citep{BEIS2018EnergyTrends}.
Northern Ireland usually imports electricity from Scotland but was a net exporter to Scotland for the first time in 2016, which continued in 2017 \citep{BEIS2018EnergyTrends}.
Scotland's annual demand is calculated by subtracting the exported electricity from and adding the imported electricity to its total generation.
This gives Scotland a current typical electricity demand of 35,810~GWh.

% Please add the following required packages to your document preamble:
% \usepackage{booktabs}
\begin{table}[htbp]
	\caption{Generation, exports, imports and demand of electricity in Scotland in 2017 \citep{BEIS2018ElecUK}.}
	\label{tbl:elec_demand}
	\centering
	\begin{tabular}{@{}lr@{}}
		\toprule
		Electricity in Scotland 2017 & GWh \\ \midrule
		Total generated & 48,678 \\
		Electricity exported to England & -13,013 \\
		Electricity imported from Northern Ireland & 145 \\ \midrule
		Demand (sum of above) & 35,810 \\ \bottomrule
	\end{tabular}
\end{table}



%%% SUBSECTION %%%

\subsection{Upscale of renewable generation mix}

Tables~\ref{tbl:installed_cap}, \ref{tbl:RE_gen} and \ref{tbl:days} respectively present the installed capacity of renewable technologies in Scotland, the renewable generation in Scotland and the number of days per year from 2012 or 2013 to 2017.
Equation~\ref{eq:CF} represents the annual capacity factor (CF), i.e. the ratio of annual energy yield over the maximum possible annual energy yield.
The capacity factors for Scotland's renewable technologies were calculated per year by inputting the data in the aforementioned tables in Equation~\ref{eq:CF}.
Equation~\ref{eq:CF_example} demonstrates this with the calculation of the capacity factor for onshore wind energy in 2013.
The capacity factors and their averages are presented in Table~\ref{tbl:CFs}.

% Please add the following required packages to your document preamble:
% \usepackage{booktabs}
\begin{table}[H]
	\caption{Cumulative installed capacity of renewable technologies in Scotland from 2012 to 2017 \citep{BEIS2018REs}.}
	\label{tbl:installed_cap}
	\centering
	\begin{tabular}{@{}lrrrrrr@{}}
		\toprule
		Cumulative installed capacity (MW) & 2012 & 2013 & 2014 & 2015 & 2016 & 2017 \\ \midrule
		Onshore Wind & 3,765 & 4,589 & 5,079 & 5,398 & 6,327 & 7,389 \\
		Offshore Wind & 190 & 190 & 197 & 187 & 187 & 246 \\
		Shoreline wave / tidal & 7 & 7 & 7 & 8 & 13 & 18 \\
		Solar PV & 95 & 133 & 175 & 264 & 326 & 323 \\
		Small scale Hydro & 158 & 171 & 189 & 232 & 290 & 313 \\
		Large scale Hydro & 1,339 & 1,339 & 1,339 & 1,339 & 1,339 & 1,341 \\
		Landfill gas & 115 & 115 & 116 & 116 & 116 & 116 \\
		Sewage sludge digestion & 9 & 7 & 7 & 7 & 7 & 7 \\
		Other biomass & 138 & 150 & 230 & 236 & 259 & 295 \\ \bottomrule
	\end{tabular}
\end{table}

% Please add the following required packages to your document preamble:
% \usepackage{booktabs}
\begin{table}[H]
	\caption{RE generation in Scotland from 2013 to 2017 \citep{BEIS2018REs}.}
	\label{tbl:RE_gen}
	\centering
	\begin{tabular}{@{}lrrrrr@{}}
		\toprule
		RE Generation (GWh) & 2013 & 2014 & 2015 & 2016 & 2017 \\ \midrule
		Onshore Wind & 10,563 & 11,130 & 13,340 & 11,954 & 16,448 \\
		Offshore Wind & 587 & 569 & 539 & 502 & 616 \\
		Shoreline wave / tidal & 1 & 2 & 2 & 0 & 4 \\
		Solar PV & 96 & 143 & 186 & 276 & 290 \\
		Hydro & 4,369 & 5,484 & 5,814 & 5,149 & 5,356 \\
		Landfill gas & 563 & 533 & 503 & 493 & 445 \\
		Sewage sludge digestion & 31 & 28 & 26 & 32 & 36 \\
		Other biomass (inc. co-firing) & 778 & 1,155 & 1,334 & 1,374 & 1,971 \\ \bottomrule
	\end{tabular}
\end{table}

% Please add the following required packages to your document preamble:
% \usepackage{booktabs}
\begin{table}[H]
	\caption{Number of days in the years from 2013 to 2017 \citep{BEIS2018REs}.}
	\label{tbl:days}
	\centering
	\begin{tabular}{@{}lrrrrr@{}}
		\toprule
		Year & 2013 & 2014 & 2015 & 2016 & 2017 \\ \midrule
		Days & 365 & 365 & 365 & 366 & 365 \\ \bottomrule
	\end{tabular}
\end{table}

\input{tables/CFs}

	\begin{equation}\label{eq:CF}
		\begin{split}
			CF & = \frac{annual\;energy\;yield\;[MWh]}{maximum\;possible\;energy\;yield\;[MWh]} \\
			& \\
			& = \frac{generation\;[MWh]}{average\;capacity\;[MW] \times hours\;in\;the\;year\;[h]}
		\end{split}
	\end{equation}

	\begin{equation}\label{eq:CF_example}
		\begin{split}
			CF & = \frac{2013\;generation\;[GWh] \times 1000}{\frac{2012\;capacity\;+\;2013\;capacity}{2}\;[MW] \times 24\;[h] \times days\;in\;2013} \\
			& \\
			& = \frac{10,563\;[GWh] \times 1000}{\frac{3,765\;+\;4,589}{2}\;[MW] \times 24\;[h] \times 365} = \frac{10,563,000\;[MWh]}{4,177\;[MW] \times 8,760\;[h]} \\
			& \\
			& = 29\%
		\end{split}
	\end{equation}

\textbf{Calculate typical generation based on 2017 installed capacity and average CFs.}
\textbf{Upscale installed capacity so generation meets demand of 35,810~GWh.}

% Please add the following required packages to your document preamble:
% \usepackage{booktabs}
\begin{table}[htbp]
	\caption{.}
	\label{tbl:upscale}
	\centering
	\begin{tabu} to \textwidth {@{}lX[r]X[r]X[r]lX[r]X[r]@{}}
		\toprule
		\multicolumn{1}{r}{} & \multicolumn{1}{r}{\textbf{2012-2017}} & \multicolumn{1}{r}{\textbf{2017}} & \multicolumn{1}{r}{\textbf{Typical}} &  & \multicolumn{2}{r}{\textbf{Scenario: Scale up 36.4\%}} \\
		& Average Capacity Factors & Cumulative Installed Capacity (MW) & RE Generation (GWh) &  & Cumulative Installed Capacity (MW) & RE Generation (GWh) \\ \midrule
		Onshore Wind & 27\% & 7,389 & 17,452 &  & 10,079 & 23,805 \\
		Offshore Wind & 33\% & 246 & 707 &  & 336 & 964 \\
		Shoreline wave / tidal & 2\% & 18 & 3 &  & 25 & 5 \\
		Solar PV & 10\% & 323 & 287 &  & 441 & 392 \\
		Small scale Hydro & 38\% & 313 & 1,047 &  & 427 & 1,429 \\
		Large scale Hydro & 38\% & 1,341 & 4,487 &  & 1,829 & 6,121 \\
		Landfill gas & 50\% & 116 & 508 &  & 158 & 693 \\
		Sewage sludge digestion & 48\% & 7 & 29 &  & 10 & 40 \\
		Other biomass & 68\% & 295 & 1,762 &  & 402 & 2,403 \\ \midrule
		Total &  & 10,049 & 26,283 &  & 13,705 & 35,851 \\ \midrule
		\multicolumn{3}{l}{Percentage of typical demand (35,810 GWh)} & 73.4\% &  &  & 100.1\% \\ \bottomrule
	\end{tabu}
\end{table}



%%% SUBSECTION %%%

\subsection{Energy storage power capacity}

\ldots



%%% SUBSECTION %%%

\subsection{Energy storage capacity OR demand}